\author{\huge Zaeem Nalla - 1178454}
\documentclass{article}
\usepackage[a4paper, margin=0.75in, total={6in, 10in}]{geometry}
\usepackage{hyperref}
\usepackage{enumerate}
\usepackage[shortlabels]{enumitem}
\usepackage{multirow}
\usepackage{tabularx}
\usepackage{graphicx}

\hypersetup{
	colorlinks=true,
	linkcolor=black,
	filecolor=blue,
	urlcolor=blue,
}
\setlength{\parindent}{0em}
\setlength{\parskip}{1em}

\begin{document}
\begin{titlepage}
\title{\textbf{\Huge Software Requirements Specification}}
\date{\large 28 August, 2017}
\bigskip
\bigskip
\maketitle
\thispagestyle{empty}
\centerline{\includegraphics[scale=0.65]{AndromedaLogo}}

\begin{table}
\Large
\begin{center}
\begin{tabular}{|m{5cm}|m{5cm}|}
\hline
\multicolumn{2}{|m{10cm}|}{\textbf{Team Members}\centering} \\
\hline
\hline
Harvey Muyangayanga & 941446\\
Jason Parry & 1046955\\
Uvir Bhagirathi & 1141886\\
Zaeem Nalla & 1178454\\
\hline
\end{tabular}
\end{center}
\end{table}

\end{titlepage}
\section{Executive Summary}

The Software Requirements Specification document defines the requirements of the proposed system to be designed by Andromeda for the SchedEx project - scheduling examinations for various college/university courses in various venues and time slots - and recognises each requirement as either Functional (sub-categorised into Mandatory or Optional) or Non-functional.

The software solution will serve as a portal for all the identified users: students, lectures, course coordinators and administrators. It will create an optimal exam timetable and then provide them with relevant information and features based on this such as personalised timetables, Google calendar event creation, exam details and navigation to the venue.

The Formal Requirements Definition tabulates the requirements under key categories: Interfaces, Functional Capabilities, Performance Levels, Reliability, Security/Privacy and Quality. This classifies the functionalities that the system should consist of accordingly and in a manner that assists the development team during the design phase of the system. Thereafter constraints of the project are listed as a consideration of pseudo-requirements that should be respected and adhered to.

A useful conclusion which can be compiled from the list of functional requirements is a Use Case Set consisting of use cases grouped by the business features to which they are related i.e. Core (e.g. generating a new timetable), Maintenance (e.g. updating course preferences), Reporting (e.g. a report of clashes - if any exist) and Archiving (e.g. storing old timetables). This shows the actions that needs to be carried out and that the system should support in order for the functional requirements to be captured.

\section{Project Scope}

The project scope is based around implementing a software system to schedule class-examination to examination venues. The system will take into consideration details such as class size, preferred examination date and time when scheduling the examination and venue. Once a examination timetable is generated students and lecturers will be provided with a personalised version of the timetable consisting of just the courses he/she is associated with. 

The purpose of this project is to provide a robust examination scheduling software to be used by tertiary education facilities with minimum examination clashes. If an examination clash occurs then the relevant personnel will be notified and will be corrected manually.

This project contributes to the need for an examination scheduling solution that's capable of generating an optimal timetable with minimal clashes and as accommodating as possible for the various course combinations for students and lecturers alike.

The purpose of the project (i.e. exam scheduling) has the expectation of ensuring that each set of exams at a tertiary (or other similar) institution has a minimal amount of clashes and fair spread of a degree's exams across the examination period (i.e. a student won't have a situation where they'll have to write all 4 of their exams in a week given the examination period is four weeks)

The goals of the project include the creation of a web application that optimises the exam scheduling process and reports to the system actors with meaningful information, creating the system in such a manner that it is aligned with the client's requirements (functionality), making a robust and reliable system, meeting phase deadlines, planning thoroughly according to the SDLC and operating cohesively as an organisation to provide the requested product.

The deliverables of the project include the fully functional web application in the allocated time frame, documents detailing the various phases of the project (including iterations) which will provide the foundation for the final product, presentations which will provide user-centric information to the clients and proof of concept and feasibility of the project, detailing and implementing the many requirements of the project (including but not limited to the mechanical data processing requirements i.e. identifying and getting the business processes and work-flows up and running, and reporting requirements) which will lead to a successful system.

The project is time sensitive and, as such, should be completed before the required deadline. This will be achieved by using an Agile approach to the Software Development Life Cycle, specifically, the SCRUM methodology. This will ensure that each phase of the project gets the required attention leading to a project of high standards which is complete and considered.

Andromeda hopes to provide a robust platform that will ease the process of exam scheduling and be a useful solution for tertiary institutions.
\end{document}